\documentclass[a4paper]{article}


%% Sets page size and margins
\usepackage[a4paper,top=3cm,bottom=2cm,left=3cm,right=3cm,marginparwidth=1.75cm]{geometry}
\usepackage{amsmath}

\title{Algebra Problems}
\author{Pradipta Bora}

\begin{document}
\maketitle


\section{Introduction}

This document contains problems of algebra, combinatorics and precalculus of medium - hard difficulty level. Questions have been collected from various sources.

\section{Problem Set}

\begin{enumerate}
\item Find the value of $$ \sum_{k=1}^{\infty} (\log_e 2)^k \sum_{r=1}^k \frac {r^2}{r!(k-r)!} $$

\item Find $$ \sum_{k=0}^{\infty} \sum_{j=0}^{\infty} \sum_{i=0}^{\infty} \frac{1}{3^i3^j3^k}$$ with the condition $ i\neq j \neq k$.

\item If $ 9^x + 2(1-a) 3^x + a = 0 $ has two roots $p, q$ such that $pq < 0$ where $a$ is an
integer from $0$ to $30$ (inclusive), then find the number of values of $a$.

\item Find the value of $$ \dbinom{n}{1} - (1 + \frac{1}{2})\dbinom{n}{2} + (1 + \frac{1}{2} + \frac{1}{3})\dbinom{n}{3}... (-1)^{n-1}(1 + \frac{1}{2} ... + \frac{1}{n})\dbinom{n}{n}$$ in terms of $n$. 

\item If $z^{17} = 1$ where $ z \neq 1$ then find the value of $$ \sum_{k=1}^{k=16} \frac{z^{2k} - 1}{z^{2k} + z^{k} + 1} $$

\item Let $f(n)$ be the number of all permutations $(a_1,a_2, \cdots , a_n)$ of $(1,2, \cdots ,n)$ such that $a_i > a_{i+1} $ for exactly one index $i$ , $ 1 \le i \le n-1$. Find the last digit of $f(10)$.
\item Two $n$ digit numbers $a, b$ are said to be compatible if their digits are permutations of each other. Also $a, b$ can have leading zeroes. For example $1203$ and $0213$ are $4$ digit compatible numbers. Find the size of the largest possible set of $6$ digit numbers (leading zeroes allowed) such that no two numbers of this set are compatible.

\item Let $A = \lbrace{1,2,3, \cdots , 100} \rbrace$. Find the total number of subsets $B$ of $A$ such that $B$ has 20 elements and for any $b_1,b_2 \in B$ , $|b_1 - b_2| \ge 4$.

\item Find the value of:
$$ \sum_{n=1}^{\infty}\frac{\cos(nx)} {3^n} $$ if $x = \frac{\pi}{6}$

\item Consider the set S = $\lbrace{
1, 2, 3 \cdots n} \rbrace$. A subset of this set is called arithmetic if the elements of this subset are in arithmetic progression when written in ascending order. 

Such a subset is called maximal arithmetic if we cannot increase the size of this subset by adding another element from S. Let $f(n)$ be the number of such maximal arithmetic subsets for a fixed n.

Find the remainder when $f(25)$ is divided by $9$.

\item Consider a Polynomial $P(x)$ with real coefficients with the property that whenever $P(x) \in Z $ then $ x \in Z $. Also $P(0) = 2017$. If the maximum possible degree of $P(x)$ is $n$ then what is the remainder when $n$ is divided by 10?
\end{enumerate}


\end{document}